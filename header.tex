% -*- coding: utf-8 -*-

\documentclass[a4peper, 11pt, french]{article}

%- Encodage langue et font
% Typographie française. et indentation de la première ligne des paragraphes.
\usepackage{babel, indentfirst}
% Encodage UTF-8
\usepackage[utf8]{inputenc}

%-- Font
% Mode d'encodage de la police
% T1 
% T1 est le nom LaTeX pour Fontes extended computer modern (EC). (codage des caractères dit « de Cork.
% Dans ce cas LATEX fera appel, par défaut, aux fontes ec pour le mode texte et aux fontes cm pour le mode mathématique. 
\usepackage[T1]{fontenc}

% French Cursive, pour le Texte
%\usepackage[default]{frcursive}
% police Euler pour les mathématiques.
%\usepackage{euler}

% Affichage correct des caractères diacritiqués français.
%\usepackage{lmodern}
\usepackage{lmodern,tgpagella}

%- inclure des images
\usepackage{graphicx}

%- Math
\usepackage{amsmath}
\usepackage{amssymb}
% siunitx pour formater les nombres avec des abréviation en plus.

\usepackage[load-configurations = abbreviations]{siunitx}
\sisetup{locale = FR, detect-all} % L’option « detect-all » permettra de garder la même fonte que dans le texte.


%- Mise en page
% Largeur du texte.
\textwidth 17cm
% Hauteur du texte.
\textheight 25cm

%-- Définition des marges
% Marge supplémentaire au dessus de l'entête.
%\topmargin 

% Marge supplémentaire dans les pages impaires d'un document recto-verso 
% ou pour toutes les pages dans le cas d'un document recto uniquement.
%\oddsidemargin -0.24cm

% Marge supplémentaire dans la marge gauche des pages paires des documents recto-verso.
%\evensidemargin -1.24cm 

%-- Entête et Bas de page
% Hauteur de l'entête.
%\headheight 
%\headheight -1.5cm % Version Ensimag

% Distance entre la dernière ligne d'entête et la première ligne du corps du document.
%\headsep 

%\topskip 0cm

% Distance entre la dernière ligne du texte et la première ligne du bas de page.
%\footskip 

%% Note de marge
% Espace vertical minimum entre deux notes de marge.
%\marginparpush 

% Marge horizontal entre le corps du document et les notes de marge.
%\marginparsep 

% Llargeur des notes de marge.
%\marginparwidth 

